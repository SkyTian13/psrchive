\documentclass[12pt]{article}
\usepackage{amsfonts}

% bold math italic font
\newcommand{\mbf}[1]{\mbox{\boldmath $#1$}}
\newcommand{\mbfs}[1]{\mbox{\scriptsize\boldmath $#1$}}

\newcommand{\trace}{{\rm tr}}
\newcommand{\pauli}[1]{\ensuremath{ {\mbf{\sigma}}_{#1} }}

\begin{document}

\section{Introduction}

This memo considers the problem of searching for the best fit between
two polarimetric pulse profiles in the Fourier domain.  In practise,
one of these will be a standard template with high signal-to-noise
ratio and the other will be a new observation that is to be fitted to
the standard.  From this fit will be derived the longitudinal shift as
well as the polarimetric transformation between the two profiles.

Let $\mbf{\rho}^\prime (\phi_n)$ represent the model polarization as
function of discrete pulse longitude, $\phi_n$, where $0\le n\le N-1$.
The model is related to the standard, $\mbf{\rho}_0(\phi_n)$, by
\begin{equation}
\mbf{\rho}^\prime (\phi_n) = \mbf{\rho}_{\mathrm DC} + {\bf J} \mbf{\rho}_0
  (\phi_n - \varphi) {\bf J}^\dagger + \mbf{\rho}_{\mathrm N}(\phi_n)
\end{equation}
where $\mbf{\rho}_{\mathrm DC}$ represents a DC offset between the two
profiles, ${\bf J}$ is the Jones matrix representing the polarimetric
transformation between the observed and standard profiles, $\varphi$
is the longitudinal shift between the profiles, and
$\mbf{\rho}_{\mathrm N}$ is the coherency matrix of the system noise.

Now define the Discrete Fourier Transform (DFT),
\begin{equation}
\mbf{\rho}(\Phi_m) = \sum_{n=0}^{N-1} \mbf{\rho}(\phi_n) e^{-i2\pi mn/N}
\end{equation}
and take the DFT of equation (1) to yield
$\mbf{\rho}^\prime (\Phi_0) =  N\mbf{\rho}_{\mathrm DC}$ and
\begin{equation}
\mbf{\rho}^\prime (\Phi_m) =  {\bf J}
  \mbf{\rho}_0 (\Phi_m) e^{-i2\pi m \varphi} {\bf J}^\dagger +
  \mbf{\rho}_{\mathrm N}(\Phi_m); \; 1\le m\le N-1
\label{eqn:fourier_rho}
\end{equation}

%The noise at each frequency, $\mbf{\rho}_{\mathrm N}(\Phi_m)$, is
%characterized by the noise in each of the four observed Stokes paramters.

Now consider the measured Stokes parameters, $S_k^\prime(\phi_n)$, and
their DFTs, $S_k^\prime(\Phi_m)$.  The best-fit model parameters will
minimize the objective merit function,
\begin{equation}
\chi^2 = \sum_{m=1}^{M/2} \sum_{k=0}^3 
	{ \Delta S_k(\Phi_m)\Delta S_k^*(\Phi_m) \over \sigma_k^2 },
\label{eqn:merit}
\end{equation}
where $\Delta S_k(\Phi_m) = S_k^\prime(\Phi_m) -
\trace[\pauli{k}\;\mbf{\rho}^\prime(\Phi_m)]$ and $\sigma_k$
characterizes the system noise in the frequency domain of each Stokes
profile. Note that, because the Stokes parameters,
$S_k^\prime(\phi_n)$, are real ($\mbf{\rho}(\phi_n)$ is Hermitian), it
is necessary to consider only the first $M/2$ frequency channels.

To find the minimum of equation~\ref{eqn:merit} requires calculating
its partial derivatives with respect to $\varphi$ as well as the
scalar parameters that describe {\bf J}...  similar to van Straten
(2004?) ...

\end{document}
