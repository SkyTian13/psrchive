\documentclass[12pt]{article}
\usepackage{bm}       % bold mathematic symbols

% symbol used for sqrt(-1)
\newcommand{\Ci}{{\rm i}}

% general Lorentz boost and rotation
\newcommand\boost[1]{\ensuremath{ {\bf H}^{#1}_{\bm{\hat m}}(\beta) }}
\newcommand\rotat[1]{\ensuremath{ {\bf U}^{#1}_{\bm{\hat n}}(\phi) }}

\newcommand\pauli[1]{\ensuremath{ \bm{\sigma}_{\rm #1} }}

\begin{document}

\section{Introduction}

Quaternions were invented by the famous Irish mathematician, William
Rowan Hamilton, when he carved the fundamental formula of quaternion
algebra,
\begin{equation}
i^2=j^2=k^2=ijk=-1
\end{equation}
into the stone of the Brougham bridge.

An arbitrary quaternion, $\bm{U}$, is given by the linear combination
\begin{equation}
\bm{U}=u_0,+u_1i+u_2j+u_3k, \hspace{5mm} u_i\in\Re. 
\end{equation}
and the multiplicative quaternion group may be said to be isomorphic with
the $2\times2$ unitary tranformations, SU(2).

A quaternion with complex coefficients is known as a biquaternion.
The algebra of biquaternions is isomorphic to $2\times2$ complex
matrices.  With biquaternions, a Hermitian matrix, $\bm{H}$, may be 
represented by the linear combination:
\begin{equation}
\bm{H}=h_0\pauli{0} + h_1\pauli{1}
	+ h_2\pauli{2} + h_3\pauli{3}, \hspace{5mm} h_i\in\Re
\end{equation}
where $\pauli{0}$ equals the identity matrix, $\bm{I}$, and $\pauli{1-3}$
are the Pauli spin matrices:
\begin{eqnarray}
\pauli{1} = \left( \begin{array}{cc}
1 & 0 \\
0 & -1 
\end{array}\right)
&
\pauli{2} = \left( \begin{array}{cc}
0 & 1 \\
1 & 0 
\end{array}\right)
& 
\pauli{3} = \left( \begin{array}{cc}
0 & -\Ci \\
\Ci & 0
\end{array}\right).
\label{pauli}
\end{eqnarray}
A very useful notation represents the biquaternion as a complex scalar plus
a three-vector:
\begin{equation}
\bm{Q}=[q+\bm{q}] = q\pauli{0} + \bm{q\cdot\sigma},
\end{equation}
were $\bm{q}=(q_1,q_2,q_3)$ and $\bm\sigma$ is a three-vector whose
components are the Pauli spin matrices as defined above.
Using this notation, the quaternion sub-group may be represented as
\begin{equation}
\bm{U}=[u_0+i\bm{u}]= u_0\pauli{0} + i\bm{u\cdot\sigma} \hspace{5mm} u_i\in\Re
\end{equation}


From the definitions of Eq.~\ref{pauli}, it can be shown that the Pauli spin
matrices have the following properties:
\begin{equation}
\pauli{i}^2 = \bm{I}, \hspace{5mm} 
\pauli{i}\pauli{j} = -\pauli{j}\pauli{i} = \Ci\pauli{k}
\end{equation}
where $\{i,j,k\}$ is chosen from cyclic permutations of $\{1,2,3\}$.


\section{Some Algebraic Derivations}

\subsection{Quaternion Multiplication}
As with matrices, quaternion multiplication is non-commutative.  It is
useful to derive the result of the multiplication of two quaternions in some
detail. Starting with $\bm{A}=[a+\bm{a}]$ and $\bm{B}=[b+\bm{b}]$
\begin{eqnarray}\label{multiplication}
\begin{array}{ccl}
\bm{A}\bm{B} & = &
(a_0\pauli{0} + a_1\pauli{1} + a_2\pauli{2} + a_3\pauli{3})
(b_0\pauli{0} + b_1\pauli{1} + b_2\pauli{2} + b_3\pauli{3}) \\
& = & a_0b_0\pauli{0}\pauli{0} + a_0b_1\pauli{0}\pauli{1}
	+ a_0b_2\pauli{0}\pauli{2} + a_0b_2\pauli{0}\pauli{2} \\
& & + a_1b_0\pauli{1}\pauli{0} + a_1b_1\pauli{1}\pauli{1}
	+ a_1b_2\pauli{1}\pauli{2} + a_1b_2\pauli{1}\pauli{2} \\
& & + a_2b_0\pauli{2}\pauli{0} + a_2b_1\pauli{2}\pauli{1}
	+ a_2b_2\pauli{2}\pauli{2} + a_2b_2\pauli{2}\pauli{2} \\
& & + a_3b_0\pauli{3}\pauli{0} + a_3b_1\pauli{3}\pauli{1}
	+ a_3b_2\pauli{3}\pauli{2} + a_3b_2\pauli{3}\pauli{2}
\end{array}
\end{eqnarray}


\section{Isomorphism with Jones matrices}

An arbitrary $2\times2$ complex, or Jones, matrix may be represented
by its polar decomposition,
\begin{equation}
{\bf J} = J \; \boost{} \rotat{},
\label{eqn:polar_decomposition}
\end{equation}
where $J=(\det{\bf J})^{1/2}$,
\begin{eqnarray}
\boost{} &=& \exp(\bm{\sigma\cdot}\bm{\hat m}\beta)
        = [\cosh\beta,\sinh\beta\;\bm{\hat m}],
\label{eqn:boost} \\
\rotat{} &=& \exp(i\bm{\sigma\cdot}\bm{\hat n}\phi)
        = [\cos\phi,i\sin\phi\;\bm{\hat n}],
\label{eqn:rotation}
\end{eqnarray}
and $\bm{\hat m}$ and $\bm{\hat n}$ are real-valued unit 3-vectors.
The matrices, \rotat{}\ and \boost{}, are unimodular (ie. have unit
determinant) and are known as the axis-angle representation of unitary
and Hermitian transformations, respectively.

A matrix may also be represented by ${\bf J} = J \; \rotat{\prime}
\boost{\prime}$, noting that $\rotat{\prime}\ne\rotat{}$ and
$\boost{\prime}\ne\boost{}$ (owing to the non-commutivity of matrix
multiplication).  However, as shown in Section~\ref{sec:root}, the
decomposition shown in Eq.~\ref{eqn:polar_decomposition} provides greater
advantages.



\end{document}


