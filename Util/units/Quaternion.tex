\documentclass[12pt]{article}
\usepackage{amsfonts}
\usepackage{bm}

% symbol used for sqrt(-1)
\newcommand{\Ci}{{\rm i}}

\newcommand{\C}{\mathbb{C}}
\newcommand{\R}{\mathbb{R}}

\newcommand{\code}[1]{{\tt{#1}}}

\newcommand{\boost}[1][]{{\ensuremath{ {\bf H}{#1}_{\bm{\hat m}}(\beta) }}}
\newcommand{\rotat}[1][]{{\ensuremath{ {\bf U}{#1}_{\bm{\hat n}}(\phi) }}}

\newcommand{\pauli}[1]{\ensuremath{ {\bm\sigma}_{\rm #1} }}

\begin{document}

\section{Introduction}

Quaternions were invented by William Rowan Hamilton when he carved the
fundamental formula of quaternion algebra,
\begin{equation}
i^2=j^2=k^2=ijk=-1
\end{equation}
into the stone of the Brougham bridge.
An arbitrary quaternion, $\bm{U}$, is given by the linear combination,
\begin{equation}
\bm{U}=u_0+u_1i+u_2j+u_3k, \hspace{5mm} u_i\in\R
\end{equation}
and, as with matrices, quaternion multiplication is non-commutative.

A quaternion with complex coefficients is known as a biquaternion.
The algebra of biquaternions is isomorphic to the $2\times2$ complex
matrix group.  Hermitian matrices, $\bm{H}=\bm{H}^\dagger$, may be
represented by the linear combination,
\begin{equation}
\bm{H}=h_0\pauli{0}+h_1\pauli{1}+h_2\pauli{2}+h_3\pauli{3},\hspace{5mm}h_i\in\R
\end{equation}
where $\pauli{0}$ equals the identity matrix, $\bm{I}$, and $\pauli{1-3}$
are the Pauli spin matrices:
\begin{eqnarray}
\pauli{1} = \left( \begin{array}{cc}
1 & 0 \\
0 & -1 
\end{array}\right)
&
\pauli{2} = \left( \begin{array}{cc}
0 & 1 \\
1 & 0 
\end{array}\right)
& 
\pauli{3} = \left( \begin{array}{cc}
0 & -\Ci \\
\Ci & 0
\end{array}\right).
\label{eqn:pauli}
\end{eqnarray}
A very useful notation represents the biquaternion as a scalar plus
a three-vector:
\begin{equation}\label{eqn:biquaternion}
\bm{Q}=[q_0+\bm{q}] = q_0\pauli{0} + \bm{q\cdot\sigma}, \hspace{5mm} q_i\in\C
\end{equation}
were $\bm{q}=(q_1,q_2,q_3)$ and $\bm\sigma$ is a three-vector whose
components are the Pauli spin matrices as defined above.  Using this
notation, the left-handed quaternion sub-group may be represented as
\begin{equation}\label{eqn:quaternion}
\bm{U}=[u_0+i\bm{u}]= u_0\pauli{0} + \Ci\bm{u\cdot\sigma} \hspace{5mm} u_i\in\R
\end{equation}


From the definitions of Eq.~\ref{eqn:pauli}, it can be shown that the
Pauli spin matrices have the following properties:
\begin{equation}\label{eqn:properties}
\pauli{i}^2 = \bm{I}, \hspace{5mm} 
\pauli{i}\pauli{j} = -\pauli{j}\pauli{i} = \Ci\pauli{k}
\end{equation}
where $\{i,j,k\}$ is chosen from cyclic permutations of $\{1,2,3\}$.


\section{Some Algebraic Derivations}

\subsection{Biquaternion Multiplication}
As with matrices, quaternion multiplication is non-commutative.  It is
useful to derive the result of the multiplication of two quaternions
in some detail. Start with two biquaternions, $\bm{A}=[a_0+\bm{a}]$
and $\bm{B}=[b_0+\bm{b}]$, as defined in Eq.~\ref{eqn:biquaternion},
and expand to yield
\begin{eqnarray}
\begin{array}{ccl}
\bm{A}\bm{B} & = &
(a_0\pauli{0} + a_1\pauli{1} + a_2\pauli{2} + a_3\pauli{3})
(b_0\pauli{0} + b_1\pauli{1} + b_2\pauli{2} + b_3\pauli{3}) \\
& = & a_0b_0\pauli{0}\pauli{0} + a_0b_1\pauli{0}\pauli{1}
	+ a_0b_2\pauli{0}\pauli{2} + a_0b_3\pauli{0}\pauli{3} \\
& + & a_1b_0\pauli{1}\pauli{0} + a_1b_1\pauli{1}\pauli{1}
	+ a_1b_2\pauli{1}\pauli{2} + a_1b_3\pauli{1}\pauli{3} \\
& + & a_2b_0\pauli{2}\pauli{0} + a_2b_1\pauli{2}\pauli{1}
	+ a_2b_2\pauli{2}\pauli{2} + a_2b_3\pauli{2}\pauli{3} \\
& + & a_3b_0\pauli{3}\pauli{0} + a_3b_1\pauli{3}\pauli{1}
	+ a_3b_2\pauli{3}\pauli{2} + a_3b_3\pauli{3}\pauli{3}
\end{array}
\end{eqnarray}
By applying the properties shown in Eq.~\ref{eqn:properties}, this may
be reduced to
\begin{eqnarray}\label{eqn:biquaternion_multiplication}
\begin{array}{ccl}
\bm{A}\bm{B}
& = & (a_0b_0 + a_1b_1 + a_2b_2 + a_3b_3) \pauli{0} \\
& + & (a_0b_1 + a_1b_0 + \Ci a_2b_3 - \Ci a_3b_2) \pauli{1} \\
& + & (a_0b_2 - \Ci a_1b_3 + a_2b_0 + \Ci a_3b_1) \pauli{2} \\
& + & (a_0b_3 + \Ci a_1b_2 - \Ci a_2b_1 + a_3b_0) \pauli{3}	
\end{array}
\end{eqnarray}

\subsection{Quaternion Multiplication}
Equation~\ref{eqn:biquaternion_multiplication} may be used to derive
the multiplication rule for quaternions.  Assuming that
$\bm{C}=[c_0+\Ci\bm{c}]$ and $\bm{D}=[d_0+\Ci\bm{d}]$, as defined in
Eq.~\ref{eqn:quaternion}, the quaternion product,
$\bm{X}=[x_0+\Ci\bm{x}]=\bm{CD}$, is given by
\begin{equation}\label{eqn:quaternion_multiplication}
\begin{array}{rl}
x_0 & = c_0d_0 - c_1d_1 - c_2d_2 - c_3d_3 \\
x_1 & = c_0d_1 + c_1d_0 - c_2d_3 + c_3d_2 \\
x_2 & = c_0d_2 + c_1d_3 + c_2d_0 - c_3d_1 \\
x_3 & = c_0d_3 - c_1d_2 + c_2d_1 + c_3d_0
\end{array}
\end{equation}

\section{The \code{Quaternion} template class}

The \code{Quaternion} template class was designed to implement both
quaternions and biquaternions using either Hamilton's \{$i$, $j$,
$k$\} or the Pauli spin matrices as the basis.  When using the Pauli
spin matrices as a basis, it is possible to represent a Hermitian
matrix using only four real numbers.  However, as the product of two
Hermitian matrices is not Hermitian (unless the matrices commute), it
is not possible to multiply two real quaternions in the Pauli basis.

In summary, there are four ways to create a null-constructed instance
of a Quaternion object:
\begin{verbatim}
// Hamilton's quaternion (unitary basis)
Quaternion<float> q;

// Representation of Hermitian matrix (non-multiplicative)
Quaternion<float, Hermitian> h;

// Biquaternion in unitary basis
Quaternion<complex<float>,Unitary> bq;

// Biquaternion in Hermitian basis
Quaternion<complex<float>, Hermitian> bh;
\end{verbatim}
Note that the floating-point precision is specified in the first
template argument.  As Hamilton's quaternion is most common and
useful, it is the basis used by default.  However, as shown in the
above example, it may be explicitly specified in the second template
argument with \code{Unitary}.


\section{Isomorphism with Jones matrices}

An arbitrary $2\times2$ complex, or Jones, matrix may be represented
by its polar decomposition,
\begin{equation}
{\bf J} = J \; \boost \rotat,
\label{eqn:polar_decomposition}
\end{equation}
where $J=(\det{\bf J})^{1/2}$,
\begin{eqnarray}
\boost &=& \exp(\bm{\sigma\cdot}\bm{\hat m}\beta)
        = [\cosh\beta,\sinh\beta\;\bm{\hat m}],
\label{eqn:boost} \\
\rotat &=& \exp(i\bm{\sigma\cdot}\bm{\hat n}\phi)
        = [\cos\phi,i\sin\phi\;\bm{\hat n}],
\label{eqn:rotation}
\end{eqnarray}
and $\bm{\hat m}$ and $\bm{\hat n}$ are real-valued unit 3-vectors.
The matrices, \rotat\ and \boost, are unimodular (ie. have unit
determinant) and are known as the axis-angle representation of unitary
and Hermitian transformations, respectively. The same matrix may also
be represented by ${\bf J}=J\;\rotat\boost[^\prime]$, where
$\boost[^\prime]=\rotat\boost\rotat[^\dagger]$.

\subsection{Polar Decomposition}
\label{sec:polar_decomposition}

The polar decomposition of Eq.~\ref{eqn:polar_decomposition} may be 
calculated noting that
\begin{equation}
{\bf JJ}^\dagger = (J\boost\rotat)(J^*\rotat[^\dagger]\boost[^\dagger])
	= |J|^2 \boost[^2]
\end{equation}


\end{document}


